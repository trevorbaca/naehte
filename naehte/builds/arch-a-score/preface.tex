\documentclass[10pt]{article}
\usepackage[T1]{fontenc}
%\usepackage[papersize={9in, 12in}, margin=3in]{geometry}
\usepackage[papersize={9in, 12in}, margin=1in]{geometry}
\usepackage{array, booktabs, fontspec, nopageno, xltxtra, xunicode}
\usepackage{xfrac}
%\parindent=0pt
%\parskip=12pt
\setmainfont{Adobe Garamond Pro}
\begin{document}

%\begin{center}
%{\huge PREFACE}
%\end{center}
%\textit{Start again, hide the tracks, stitches in rungs, in place of seams only
stoating. Turn under, mark, tailors’ tack holds the pattern, basting like chalk
shows the way. At the end of the work only tissue --- and for tissue read
‘text’ --- in laces left open, cording drawn tight, synthetics bound in their
bales. Gone in their time the scarring has left and his hands now show only
their care. Fold upon fold she reads through the text, she quilts it in blocks
that repeat. Wales stitched in series, backstitched in bundles, the feel of the
work is its edges: edges drawn, edges mend, fingers wended, fingers close. She
finishes heart wide-bright open. Tomorrow the brilliance of another cloak.}

`Nähte' in German are stitches, and the music is about the joining together of
body, movement, color and time, over and over in rows. Differences in bow speed
--- from quick skimmings of the strings to almost completely stopped --- are
taken as primary in the piece. Transitions between categories of bow speed are
animated by quick actions of left hand — single-, double- and triple-harmonic
trills, classes of vibrato --- and also by cross-string tremoli and bow
waverings executed by the right. The relentlessly fast coordinative movements
of the cellist’s bow and body are modulated by the inventory of tempo types
that structure the music: materials given only fleeting moments early in the
piece come back later under the magnification of time and with a
correspondingly transfigured effort of choreography --- and sensitivity --- on
the part of the cellist and her movements in the music.

%\vfill
%\textbf{Nähte} was written for Ashley Walters who gave the world premiere on 16
November 2018 at ArtshareLA in Los Angeles, California.

%\vfill

\begin{center}

{\huge Playing techniques.}

\end{center}

\begin{tabular}[t]{p{3.5cm} p{5cm}}
\textbf{XFB}
    & 
    ``extremely fast bow'': very pronounced flautando;
    play with generous amounts of bow lightly skimming the string;
    resulting color almost `fluorescent';
    prolong the color with irregular retakes of the bow as necessary;
    only occurs at quiet dynamics.
    \\[3.5cm] 
\textbf{spazzolato} \newline (\textbf{spazz.}, \textbf{spz.})
    &
    `sweep' bow repeatedly up and down string in $\sfrac{1}{2}$ clt position;
    technique is reiterative.
    \\
\textbf{spazz. strett.} \textbf{$\longrightarrow$} \textbf{larg.}
    &
    spazzolato allargando: tight spazzolato becoming looser.
    \\
\textbf{spazz. larg.} \textbf{$\longrightarrow$} \textbf{strett.}
    &
    spazzolato stringendo: loose spazzolato becoming tighter.
    \\[1cm]
\textbf{trem. strett.} \textbf{$\longrightarrow$} \textbf{larg.}
    &
    tremolo allargando: tight tremolo becoming looser.
    \\
\textbf{trem. larg.} \textbf{$\longrightarrow$} \textbf{strett.}
    &
    tremolo stringendo: loose tremolo becoming tighter.
    \\
\textbf{T} & tasto (string contact point)
    \\
\textbf{P} & ponticello (string contact point)
    \\
\textbf{DZ}
    &
    ``dead zone'': last 1 cm of string closest to bridge (but not yet touching
    bridge); pitch almost (but not completely) effaced.
    \\
\textbf{OB}
    &
    ``on bridge'': bow directly on wood of bridge;
    pitch completely effaced;
    resultant sound gives snowlike whitenoise.
    \\[2cm]
\includegraphics{../_assets/xfb-trimmed.png}
    &
    ``extremely fast bow'' spanner.
    \\[0.5cm]
\includegraphics{../_assets/circle-trimmed.png}
    &
    circle bowing;
    one complete circle per notated duration.
    \\
\end{tabular}
%
\hfill
%
\begin{tabular}[t]{p{2cm} p{5cm}}
\textbf{XFB} & ``extremely fast bow'': very pronounced flautando;
    play with generous amounts of bow lightly skimming the string;
    retake bow irregularly as necessary;
    resulting color almost `fluorescent';
    only occurs at quiet dynamics.
    \\ 
\end{tabular}

\end{document}
